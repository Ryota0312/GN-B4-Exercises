\documentclass[fleqn, 14pt]{extarticlej}
\oddsidemargin=-1cm
\usepackage[dvipdfmx]{graphicx}
\usepackage{indentfirst}
\textwidth=18cm
\textheight=23cm
\topmargin=0cm
\headheight=1cm
\headsep=0cm
\footskip=1cm

\def\labelenumi{(\theenumi)}
\def\theenumii{\Alph{enumii}}
\def\theenumiii{(\alph{enumiii})}
\def\:{\makebox[1zw][l]{:}}
\usepackage{comment}
\usepackage{url}
\urlstyle{same}

\usepackage{jtygm}  % フォントに関する余計な警告を消す
\usepackage{nutils} % insertfigure, figef, tabref マクロ

\begin{document}

\begin{center}
{\Large {\bf 平成29年度GNグループB4新人研修課題 報告書}}

\end{center}

\begin{flushright}
  2017年4月20日\\
  
  乃村研究室\ 西 良太
\end{flushright}

\section{概要}
本資料は平成29年度GNグループB4新人研修課題の報告書である.本資料では,課題内容,理解できなかった部分,作成できなかった機能および自主的に作成した機能について述べる.

\section{課題内容}
RubyによるSlackBotプログラムを作成する.具体的には以下の2つを行う.

\begin{enumerate}
\item 任意の文字列を発言するプログラムの作成
\item SlackBotプログラムへの機能追加
\end{enumerate}

SlackBotの作成に用いたRubyのバージョンは2.1.5である.

\section{理解できなかった部分}
理解できなかった部分は以下の2点である.

\begin{enumerate}
\item Rackの仕組みについて.
\item bundlerの後方互換性の有無について.

  今回の開発環境ではbundler 1.14.6を使用しており,デプロイ先のHerokuではbundler 1.13.7が使用されていた.このように運用環境のbundlerのバージョンが古いときでも問題なくGemの管理が行われるのかどうかが理解できなかった.
\end{enumerate}

\section{作成できなかった機能}
作成できなかった機能を以下に示す.

\begin{enumerate}
\item 指定したOutgoing WebHooks以外からのPOSTを拒否する機能
\item 周辺施設の検索において表示結果の数を指定する機能.
\end{enumerate}

\section{自主的に作成した機能}
自主的に作成した機能を以下に示す.

\begin{enumerate}
\item 地名,建物名等から指定したキーワードに関連する施設を検索して発言する機能.また,それらの位置を示す地図画像と経路を見ることができるGoogleMapのURLを発言する機能.
\end{enumerate}

\end{document}
