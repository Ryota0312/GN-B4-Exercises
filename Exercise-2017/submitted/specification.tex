\documentclass[fleqn, 14pt]{extarticlej}
\oddsidemargin=-1cm
\usepackage[dvipdfmx]{graphicx}
\usepackage{indentfirst}
\textwidth=18cm
\textheight=23cm
\topmargin=0cm
\headheight=1cm
\headsep=0cm
\footskip=1cm

\def\labelenumi{(\theenumi)}
\def\theenumii{\Alph{enumii}}
\def\theenumiii{(\alph{enumiii})}
\newcommand*{\figref}[1]{図\ref{#1}}
\newcommand*{\tbref}[1]{表\ref{#1}}
\def\:{\makebox[1zw][l]{:}}
\usepackage{comment}
\usepackage{url}
\urlstyle{same}

\usepackage{jtygm}  % フォントに関する余計な警告を消す
\usepackage{nutils} % insertfigure, figef, tabref マクロ


\begin{document}

\begin{center}
{\Large {\bf SlackBotプログラム 仕様書}}

\end{center}

\begin{flushright}
  2017年4月20日\\
  
  乃村研究室\ 西 良太
\end{flushright}

\section{概要}
本資料は,平成29年度GNグループB4新人研修課題にて作成したSlackBotプログラムの仕様についてまとめたものである.本プログラムは以下の2つの機能をもつ.

\begin{enumerate}
\item ``「〇〇」と言って''という発言に対して,``〇〇''と発言する.
\item ``〇〇付近の△△''という発言に対して,〇〇で指定された場所から近い△△に関連する施設とその住所を3件発言する.また,〇〇の場所と施設3件にピンを立てた地図画像を発言する.加えて,それぞれの施設について〇〇からの経路を見ることができるリンクを発言する.
\end{enumerate}

\section{機能}
本プログラムはSlackでの``@NBot''から始まるユーザの発言を受信し,それに対して返信する.返信される内容は``@NBot''に続く文字列により決定される.以下に本プログラムがもつ2つの機能について述べる.

\begin{description}
\item[(機能1)]``「〇〇」と言って''という発言に対して,``〇〇''と発言する機能.
  
   この機能はユーザの``「〇〇」と言って''という発言に対して,一番外側の鈎括弧内の文字列を発言したユーザに返信する形式で発言する.
\item[(機能2)]``〇〇付近の△△''という発言に対して,〇〇で指定された場所の近くの△△に関連する施設3件の情報を発言する機能.
  
   この機能ではユーザの``〇〇付近の△△''という発言に対して,〇〇で指定された場所周辺の△△という施設について以下の3つの情報をそのユーザに返信する形式で発言する.

\begin{enumerate}
  \item 〇〇からの距離が近い△△に関連する施設3件の施設名と住所.
  \item 〇〇からそれぞれの施設までの経路を見ることができるGoogleMapへのリンク.
  \item 〇〇と検索された3件の施設にピンを立てた地図の画像. 
\end{enumerate}

 上記の情報はGoogole Maps Geocoding API,Google Places API,Googole Static Maps APIを利用して取得または作成している.また,地図画像のURLについてはGoogle URL Shortener APIを用いて短縮したものを使用する.

\end{description}

上記の(機能1)と(機能2)のどちらにも当てはまらない文字列を受信したときは,``Hi! @User''と発言する.

\section{動作環境}

\section{動作確認済み環境}
動作確認済み環境を以下の\tbref{tab:env}に示す.

\begin{table}[t]
  \begin{center}
    \caption{動作環境}
    \label{tab:env}
    \begin{tabular}{|c|l|} \hline
      \multicolumn{2}{|c|}{Herokuサーバの環境}\\ \hline \hline
      OS & Ubuntu 14.04.5 LTS\\ 
      CPU & Intel(R) Xeon(R) CPU E5-2670 v2 @ 2.50GHz\\ 
      メモリ & 64GB\\
      ソフトウェア & \\ \hline
      \multicolumn{2}{|c|}{開発環境}\\ \hline \hline
      OS & Debian GNU/Linux 8.1 64bit\\ 
      CPU & Intel(R) Core(TM) i5-4670 CPU @ 3.40GHz\\ 
      メモリ & 1GB\\ 
      ブラウザ & firefox 52.0.2\\ 
      ソフトウェア & \\ \hline
    \end{tabular}
  \end{center}
\end{table}

\section{使用方法}
本プログラムの使用方法について述べる.本プログラムはHeroku上で動作するため,Herokuへデプロイすることで実行できる.

\section{エラー処理と保証しない動作}
本プログラムにおけるエラー処理と保証しない動作について述べる.

\subsection{エラー処理}

\subsection{保証しない動作}


\end{document}
