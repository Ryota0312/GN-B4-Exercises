\documentclass[fleqn, 14pt]{extarticlej}
\oddsidemargin=-1cm
\usepackage[dvipdfmx]{graphicx}
\usepackage{indentfirst}
\textwidth=18cm
\textheight=23cm
\topmargin=0cm
\headheight=1cm
\headsep=0cm
\footskip=1cm

\def\labelenumi{(\theenumi)}
\def\theenumii{\Alph{enumii}}
\def\theenumiii{(\alph{enumiii})}
\def\:{\makebox[1zw][l]{:}}
\usepackage{comment}
\usepackage{url}
\urlstyle{same}

\usepackage{jtygm}  % フォントに関する余計な警告を消す
\usepackage{nutils} % insertfigure, figef, tabref マクロ


\begin{document}

\begin{center}
{\Large {\bf SlackBotプログラム 仕様書}}

\end{center}

\begin{flushright}
  2017年4月20日\\
  
  乃村研究室\ 西\ 良太
\end{flushright}

\section{概要}
本資料は,平成29年度GNグループB4新人研修課題にて作成したSlackBotプログラムの仕様についてまとめたものである.本プログラムは以下の2つの機能をもつ.

\begin{enumerate}
\item ``「〇〇」と言って''という発言に対して,``〇〇''と発言する機能
\item ``〇〇付近の△△''という発言に対して,〇〇で指定された場所の近くの△△に関連する施設3件の情報を発言する機能
\end{enumerate}

\section{対象とする利用者}
本プログラムは以下のアカウントを所有する利用者を対象としている.

\begin{enumerate}
\item Slackアカウント
\item Googleアカウント
\end{enumerate}

Googleアカウントは本プログラムで使用するAPIのキー取得に必要である.

\section{機能}
本プログラムはSlackでの``@NBot''から始まるユーザの発言を受信し,それに対して返信する.返信される内容は``@NBot''に続く文字列により決定される.以下に本プログラムがもつ2つの機能について述べる.

\begin{description}
\item[(機能1)]``「〇〇」と言って''という発言に対して,``〇〇''と発言する機能
  
   この機能はユーザの``「〇〇」と言って''という発言に対して,一番外側の鈎括弧内の文字列を発言したユーザに返信する.
\item[(機能2)]``〇〇付近の△△''という発言に対して,〇〇で指定された場所の近くの△△に関連する施設3件の情報を発言する機能
  
   この機能はユーザの``〇〇付近の△△''という発言に対して,〇〇で指定された場所周辺の△△という施設について以下の3つの情報をそのユーザに返信する.

\begin{enumerate}
  \item 〇〇からの距離が近い△△に関連する施設3件の施設名と住所.
  \item 〇〇からそれぞれの施設までの経路を見ることができるGoogleMapへのリンク.
  \item 〇〇と検索された3件の施設にピンを立てた地図の画像. 
\end{enumerate}

 上記の情報はGoogole Maps Geocoding API\cite{GoogleGeocodingAPI},Google Places API\cite{GooglePlacesAPI},Googole Static Maps API\cite{GoogleStaticMapsAPI}を利用して取得または作成している.また,地図画像のURLについてはGoogle URL Shortener API\cite{GoogleURLShortenerAPI}を用いて短縮したものを使用する.

\end{description}

上記の(機能1)と(機能2)のどちらにも当てはまらない文字列を受信したときは,以下のメッセージを発言する.

\begin{verbatim}
  Hi! @ユーザー名
  Usage:"〇〇付近の〇〇", "「〇〇」と言って"
\end{verbatim}

\section{動作環境}
本プログラムはRuby 2.1.5で動作する.また,Webアプリケーションフレームワークとしてsinatra 1.4.8を利用しているためsinatraと依存関係にある\tabref{tab:sinatra_dependent}に示すGemを必要とする.


\begin{table}[t]
  \begin{center}
    \caption{sinatra1.4.8が必要とするGem}
    \label{tab:sinatra_dependent}
    \begin{tabular}{|c|c|} \hline
      Gem & バージョン \\ \hline \hline
      rack & 1.5以上2.0未満 \\
      rack-protection & 1.4以上2.0未満 \\
      tilt & 1.3以上3.0未満 \\ \hline
    \end{tabular}
  \end{center}
\end{table}

\section{動作確認済み環境}
動作確認済み環境を\tabref{tab:env}に示す.bundler以外のGemはGemfileとGemfile.lockに記述されている依存関係を用いてインストールされる.

\begin{table}[t]
  \begin{center}
    \caption{動作確認済み環境}
    \label{tab:env}
    \begin{tabular}{|c|l|} \hline
      \multicolumn{2}{|c|}{デプロイ先(Heroku)の環境}\\ \hline \hline
      OS & Ubuntu 14.04.5 LTS\\ 
      CPU & Intel(R) Xeon(R) CPU E5-2670 v2 @ 2.50GHz\\ 
      メモリ & 64GB\\
      Ruby & 2.1.5p273 \\
      Ruby Gem & bundler 1.13.7\\
      & sinatra 1.4.8 \\
      & rack 1.6.5 \\
      & rack-protection 1.5.3 \\
      & tilt 2.0.7 \\ \hline
    \end{tabular}
  \end{center}
\end{table}

\section{環境構築}
\subsection{概要}
本プログラムの動作のために必要な環境構築の項目を以下に示す.

\begin{enumerate}
  \item Herokuの設定
  \item SlackのWebHookの設定
  \item 各種Google APIのAPIキー取得
\end{enumerate}

次節でそれぞれの具体的な環境構築手順について述べる.

\subsection{具体的な手順}
\subsubsection{Herokuの設定}
\begin{enumerate}
\item 以下のURLよりHerokuにアクセスし,「Sign up」から新しいアカウントを登録する.

  \url{https://www.heroku.com/}
\item 登録したアカウントでログインし,「Getting Started with Heroku」の使用する言語として「Ruby」を選択する.
\item 「I’m ready to start」をクリックし,「Download Heroku CLI for...」からCLIをダウンロードする.
\item 以下のコマンドを実行しHerokuにログインする.

  \verb|$ heroku login|
\item 以下のコマンドを実行しHeroku上にアプリケーションを生成する.

  \verb|$ heroku create <app_name>|
\end{enumerate}
  
\subsubsection{SlackのWebHookの設定}
Slackが提供するIncoming WebhooksとOutgoing Webhooksの設定手順は以下の通りである.

\begin{description}
\item[Incoming WebHooksの設定]\
  
  \begin{enumerate}
  \item 以下のURLにアクセスする.
    
    \url{https://XXXXX.slack.com/apps/manage/custom-integrations}

    ただし,XXXXXはチーム名.
    
  \item 「Incoming WebHooks」をクリックする.
  \item 「Add Configuration」をクリックし発言先のチャンネルを選択した後「Add Incoming WebHooks integration」をクリックすることでWebHook URLを取得する.
  \item 取得したURLは以下のコマンドによりHerokuの環境変数として設定する.

    \verb|$ heroku config:set INCOMING_WEBHOOK_URL="https://XXXXXXXX"| 
  \end{enumerate}
\item[Outgoing WebHooksの設定]\
  
  \begin{enumerate}
  \item 以下のURLにアクセスする.
    
    \url{https://XXXXX.slack.com/apps/manage/custom-integrations}

    ただし,XXXXXはチーム名.
    
  \item 「Outgoing WebHooks」をクリックする.
  \item 「Add Configuration」をクリックし「Add Incoming WebHooks integration」をクリックする.ここで,「Integration Settings」の以下の項目を設定する.
    \begin{enumerate}
    \item Channelにて発言を監視するチャンネルを選択する.
    \item Trigger Word(s)にWebHookが動作する契機となる単語を設定する.
    \item URL(s)にWebHookが動作した際にPOSTを行うURLを設定する.今回はHeroku上で動作させるため以下のURLを設定する.

      \url{https://XXXXX.herokuapp.com/slack}
      
      ただし,XXXXXはHrokuに登録したアプリケーション名.
    \end{enumerate}
  \end{enumerate}
\end{description}

\subsubsection{各種Google APIのAPIキー取得}
本プログラムが使用するGoogle APIのキー取得方法について述べる.また,APIキーを取得するためにはGoogleアカウントが必要である.

\begin{description}
\item[(1)Google Maps Geocoding API]\
  
  以下のURLにアクセスし「標準 API 向けの認証」の「キーを取得する」よりAPIキーを取得する.
  
  \url{https://developers.google.com/maps/documentation/geocoding/get-api-key}\\
  \url{?hl=ja}
\item[(2)Google Places API Web Service]\
  
  以下のURLにアクセスし「標準 Google Places API Web Service を使用する場合」の「キーを取得する」よりAPIキーを取得する.
  
  \url{https://developers.google.com/places/web-service/get-api-key?hl=ja}
\item[(3)Googole Static Maps API]\
  
  以下のURLにアクセスし「標準 API 向けの認証」の「キーを取得する」よりAPIキーを取得する.
  
  \url{https://developers.google.com/maps/documentation/static-maps/get-api-key}\\
  \url{?hl=ja}
\item[(4)Google URL Shotener API]\
  
  以下のURLにアクセスし「Acquiring and using an API key」の「GET A KEY」よりAPIキーを取得する.
  
  \url{https://developers.google.com/url-shortener/v1/getting_started}
\end{description}

また,取得したAPIキーは``APIconfig.yml''という名称のファイルを用意し,以下のように記述する.
\begin{verbatim}
GooglePlaces : API key
GoogleGeocoding : API key
GoogleStaticMaps : API key
GoogleURLShortener : API key
\end{verbatim}

\section{使用方法}
本プログラムの使用方法について述べる.本プログラムはHeroku上で動作するため,Herokuへデプロイすることで実行できる.

Hrokuには以下のコマンドを用いてデプロイできる.

\begin{verbatim}
  $ git push heroku master
\end{verbatim}

\section{エラー処理と保証しない動作}
本プログラムにおけるエラー処理と保証しない動作について述べる.

\subsection{エラー処理}
本プログラムで行ったエラー処理を以下に示す.

\begin{enumerate}
\item (機能2)について,``〇〇付近の△△''というメッセージの〇〇の座標をGoogle Maps Geocoding APIが見つけられなかった場合,以下のようにユーザに返信する.
\begin{verbatim}
  @ユーザ名
  地点が特定できませんでした.
\end{verbatim}
\item (機能2)について,``〇〇付近の△△''について〇〇周辺の△△に関連する施設がGoogle Places APIによって発見できなかった場合,以下のようにユーザに返信する.
\begin{verbatim}
  @ユーザ名
  結果が見つかりませんでした.
\end{verbatim}
\end{enumerate}
\subsection{保証しない動作}
本プログラムが保証しない動作を以下に示す.

\begin{enumerate}
\item SlackのOutgoing WebHooks以外からのPOSTリクエストをブロックする動作.
\end{enumerate}

\bibliographystyle{ipsjunsrt}
\bibliography{mybibdata}

\end{document}
